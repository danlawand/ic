Neste projeto, o plano é estudar e implementar estruturas de dados e algoritmos
usados no contexto de conexidade dinâmica de grafos.  Além do funcionamento destas 
estruturas e algoritmos, serão estudados também suas análises de correção e seu 
consumo assintótico de tempo.  Ou seja, serão abordados também conceitos e técnicas
de análise de algoritmos. 

%%% Estudos preliminares
O Daniel fez um rápido estudo preliminar do problema da conexidade dinâmica em grafos, 
que deve ser aprofundado durante os próximos meses. Ele também já implementou uma 
primeira estrutura de dados, baseando-se no livro de Sedgewick e Wayne~\cite{SedgewickW2011},
que será usada como base da implementação das chamadas \emph{splay trees}~\cite{SleatorT1985}. 
Estas por sua vez são usadas na implementação das \emph{link-cut trees}. 
No presente momento, o Daniel está estudando as splay trees, 
e deve começar a implementá-las nas próximas semanas. 

%%% Objetivos específicos
O objetivo dessa iniciação científica é o aprendizado de várias estruturas de dados 
que possuem aplicações não apenas em conexidade dinâmica, mas em diversas outras 
áreas da Ciência da Computação.  Vale destacar que as estruturas de dados que serão 
abordadas são não triviais e, sendo este um projeto de iniciação científica, é 
possível que apenas parte do material mencionado seja completamente estudado e
implementado. 

%%% Objetivo secundário
Como subproduto da iniciação científica, o Daniel deve também produzir
um texto com tudo o que foi estudado. 
Splay tree é uma árvore binária de busca (ABB) com a propriedade de ser autoajustável, em que o último elemento acessado se torna a raiz da árvore~\cite{SleatorT1985}. Existem diferentes tipos de implementação, mas para o propósito deste trabalho utilizaremos a seguinte interface:  

% \begin{lstlisting}
% // splay.h
% void  splay(Node);
% void  printSPLAY(Node, int);
% Node* split(Node);
% Node  maximum(Node);
% Node  minimum(Node);
% void  pushBitDown(Node);
% \end{lstlisting}

% adicionar uma descrição do que cada rotina faz, e quanto tempo consome. 


\begin{verbatim}
// splay.h
void  splay(Node);
Node* split(Node);
Node  maximum(Node);
Node  minimum(Node);
\end{verbatim}
% Esconde por enquanto void  pushBitDown(Node);

Note que os nós (\textit{Node}) não tem campo chave (\textit{Key}), tratando-se de uma ABB com chaves implícitas, ou seja, não utilizamos as chaves para realizar qualquer tipo de operação. Em uma splay tree tradicional, as chaves da sub-árvore esquerda são menores do que a de seu pai e as chaves da sub-árvore direita são maiores do que a de seu pai, e para encontrar determinado nó, passamos a chave como argumento. No nosso caso, as rotinas que usamos na link-cut-tree, se utilizam diretamente dos ponteiros aos nós, descartando a utilidade de realizar operações de buscas pela chave para encontrar aos nós, por isso nossa splay tree possui chaves implícitas. Além disso, trabalhamos com a profundidade dos nós, ou seja, em relação ao pai do nó, os nós que estão à esquerda são menos profundos (mais superficiais) e os nós que estão à direita são mais profundos.  

%  Nessa implementação, NÃO HÁ INSERÇÃO, POIS NESSAS SPLAYS ELAS CRESCEM COM AS ROTINAS JOIN E DIMINUEM ATRAVES DA ROTINA SPLIT QUE QUEBRA ARVORE SPLAY EM DUAS.
% A ROTINA JOIN RECEBE DUAS SPLAYS ...
% SPLIT...
% MakeSPLAY...
% MAXIMUM...
% MINIMUM...
% Inverte ... reflete a splay, isso com chaves explícitas não daria certo


Nessa implementação, as inserções dos nós em uma splay tree ocorrem de maneira diferente do que a tradicional, sem necessidade de usar chaves. Fizemos com que operação de inserção ocorresse na rotina \textit{link} das link-cut tree, mas para mostrar a consistência da operação descreveremos brevemente como a inserção funciona na splay tree do nosso caso.  

% Seja \textbf{\textit{v}} o nó raiz de uma splay tree e \textbf{\textit{w}} um nó qualquer sem filho esquerdo. Queremos tornar \textbf{\textit{w}} mais profundo do que \textbf{\textit{v}}. 

Seja o nó \textbf{\textit{v}} raiz de uma splay tree e \textbf{\textit{w}} um nó qualquer sem filhos esquerdos, queremos tornar \textbf{\textit{w}} mais profundo do que \textbf{\textit{v}}. Para isso, faremos com que \textbf{\textit{v}} seja filho esquerdo de \textbf{\textit{w}}. Desta forma, como os filhos esquerdos são menos profundos que seu pai, preservamos a propriedade de que \textbf{\textit{v}} e seus filhos sejam menos profundos do que \textbf{\textit{w}}.  

A rotina \textit{splay} é a rotina principal que nos fez escolher as splay tree como árvores auxiliares para construir as link-cut tree, pois, como veremos a frente, ela auxilia na rotina \textit{access} que é a base para as operações \textit{link} e \textit{cut} da link-cut tree.  

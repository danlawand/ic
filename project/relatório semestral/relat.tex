% Da página da FAPESP: 
%
% O projeto de pesquisa deve ser apresentado de maneira clara e
% resumida, ocupando no máximo 20 páginas datilografadas em espaço
% duplo. Para propostas encaminhadas através do Sistema de Apoio a
% Gestão (SAGe), deve anexar documento tipo DOC ou PDF de até 5Mb.
% 
% Deve compreender:
% 
%     * Resumo (máximo 20 linhas);
%     * Introdução e justificativa, com síntese da bibliografia fundamental;
%     * Objetivos;
%     * Plano de trabalho e cronograma de sua execução;
%     * Material e métodos;
%     * Forma de análise dos resultados. 

\documentclass[12pt]{article}
\usepackage[utf8]{inputenc}
\usepackage[brazil]{babel}
\usepackage{import}
\usepackage{verbatim}

\usepackage{listings}
\usepackage{color}

\definecolor{dkgreen}{rgb}{0,0.6,0}
\definecolor{gray}{rgb}{0.5,0.5,0.5}
\definecolor{mauve}{rgb}{0.58,0,0.82}
%% Floating package
\usepackage{floatflt,epsfig,epsf}
\usepackage[dvipsnames]{xcolor}

\lstset{
  language=C,
  basicstyle=\footnotesize,
  numbers=left,
  numberstyle=\tiny\color{gray},
  stepnumber=1,
  numbersep=5pt,
  backgroundcolor=\color{white},
  showspaces=false,
  showstringspaces=false,
  showtabs=false,
  frame=none,
  tabsize=2,
  captionpos=b,
  breaklines=true,
  breakatwhitespace=false,
  title=\lstname,
  keywordstyle=\color{blue},
  commentstyle=\color{dkgreen},
  stringstyle=\color{mauve},
}


\newtheorem{defini}{Definição}[section]
\newtheorem{prob}{Problema}[section] %[defini]
\newcommand{\Oh}{\mathrm{O}}

% \setlength{\parskip}{0.1cm}

\sloppy

\begin{document}
\begin{center}

{\Large {\bf Conexidade Dinâmica}} 

{\large {\em Projeto de Pesquisa para Iniciação Científica}

%\footnote{ Este pedido está inserido no Projeto
%     \textsc{Aspectos Estruturais e Algorítmicos de Objetos
%        Combinatórios} (Proc. FAPESP no. 96/04505--2), sob
%      coordenação do professor Yoshiharu Kohayakawa.  }
}

\vspace{0.2cm}
{\small 
{\bf Orientadora:} Cristina Gomes Fernandes \\
{\bf Aluno:} Daniel Angelo Esteves Lawand
}

\vspace{5mm} 

{\small Este projeto de pesquisa acompanha a requisição de bolsa de \\ 
  Iniciação Científica para o aluno Daniel Angelo Esteves Lawand.}

\end{center}

\section{Introdução}  
\import{./}{introducao.tex}

\section{Splay trees}  
\import{./}{splay.tex}

\section{Link-cut trees}  

\section{Justificativa}  
\import{./}{justificativa.tex}

\section{Objetivos}  
\import{./}{objetivos.tex}


\section{Plano de trabalho e cronograma}  

O Daniel começou recentemente a trabalhar nesse projeto. Numa primeira
fase, ele estudou a seção~1.1 do capítulo \emph{Dynamic Graphs} no livro de 
Demetrescu et al.~\cite{DemetrescuFI2004} sobre os conceitos básicos de grafos 
dinâmicos e as suas possíveis estruturas de dados para a implementação. 
Após isto, estudou o funcionamento das link-cut trees~\cite{DemaineHJSI2012}, 
implementou uma Árvore Binária de Busca e está estudando as splay trees.

%%% Estudos iniciais 
A nossa intenção é
inicialmente estudar e implementar os algoritmos e as estruturas de dados que são base para as link-cut trees. 
Tendo passado pelos conceitos fundamentais, inicia-se a implementação de tal estrutura.
Junto a isso, se faz a análise de desempenho dos algoritmos envolvidos, 
delimitando o seu consumo assintótico de tempo.
Ao término desta primeira etapa, o Daniel irá elaborar um texto descrevendo o que foi necessário para a implementação das link-cut trees e expor os resultados obtidos das análises.
Estimamos que levaremos cerca de 4 meses para concluir esta etapa. 

%%% Início da escrita do texto
%%% Estudos seguintes 
Ao passo que escreve o texto referente à primeira etapa, ele seguirá para o próximo tópico, que é o estudo e implementação das Euler tour trees. Primeiramente, ele irá estudar o funcionamento da estrutura de dados e verficar se esta faz uso de outras estruturas, caso positivo, ele irá estudar e implementar as estruturas de dados base para as Euler tour trees. Ao término deste estudo, o Daniel iniciará a implementação e análise das Euler tour trees. 

%%% Estudos seguintes 
Ao fim desta segunda etapa, o Daniel escreverá um texto descrevendo como foi a implementação das Euler tour trees e os resultados obtidos das análises. A partir disso, ele irá escrever um relatório intermediário contendo as informações dos textos já escritos e irá comparar o comportamento de cada estrutura de dados.  Esperamos que esta etapa leve aproximadamente o mesmo tempo que a primeira fase do projeto, ou seja, cerca de 4 meses, embora ainda não tenhamos estudado esta estrutura de dados para ter uma noção mais concreta sobre esta previsão.  

%%% Muito material
Se conseguirmos seguir este cronograma inicial, teremos ainda tempo para estudar mais um tópico relacionado ao tema.  Como há bastante material disponível, pretendemos, a medida que nos aprofundamos nos estudos planejados para a primeira e segunda etapas, escolher mais algum tópico relacionado para estudar, se houver tempo após completarmos estas primeiras etapas do projeto.

%%% Possível desvio de rota... 

%%% Ainda sobre o texto

%%% Cronograma



\section{Material e métodos}  

Há muito material sobre grafos dinâmicos na literatura.
Primeiramente, utilizamos o livro de Demetrescu, Finnochi e Italiano~\cite{DemetrescuFI2004} 
para ter uma familiaridade com os conceitos de grafos dinâmicos. 
Utilizaremos as notas de aula do Professor Demaine~\cite{DemaineHJSI2012, DemaineL2007}, do MIT, 
e os artigos de Sleator e Tarjan~\cite{SleatorT1983} e de Henzinger e King~\cite{HenzingerK1995} 
para nos aprofundarmos nas \emph{link-cut trees} e nas \emph{Euler tour trees}. 
Utilizaremos o livro de Sedgewick e Wayne~\cite{SedgewickW2011} e um outro artigo de 
Sleator e Tarjan~\cite{SleatorT1985} para estudar as estruturas de dados que são base 
para implementação das estruturas de grafos dinâmicos.

O método que usaremos para conduzir essa iniciação científica é tradicional. 
O aluno estudará cuidadosamente os diversos resultados, e irá implementando 
parte do que for estudado, e teremos reuniões a cada duas semanas para discutir 
os assuntos estudados e as implementações.  Ao mesmo tempo que estuda e implementa
parte dos tópicos, o aluno escreverá um texto, o que possibilitará uma melhor 
avaliação de quão bem o material estudado está sendo absorvido.

\section{Forma e análise dos resultados}  

Durante todo o período de estudo, além das implementações, o aluno estará preparando 
um texto, que, ao final do trabalho, conterá tudo que foi estudado na iniciação científica. 
Este é o principal objeto que pode ser usado na análise do trabalho que estará sendo desenvolvido.
Fora isso, evidentemente esperamos que o aluno mantenha o bom desempenho (ou até melhore) no BCC. 

\bibliographystyle{plain}
\bibliography{relat}

\end{document}

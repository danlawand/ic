O estudo de estruturas de dados e algoritmos mais sofisticados ou com análises
mais complexas proporciona uma oportunidade de aprofundamento de conhecimentos 
importantes adquiridos durante um bom curso de graduação em Ciência da Computação.

Problemas dinâmicos em grafos dependem de estruturas de dados não triviais, para 
serem resolvidos de maneira eficiente.  Várias destas estruturas não são abordadas 
nas disciplinas obrigatórias de uma graduação em Ciência da Computação. 

Redes como a internet são extremamente dinâmicas, e ampliaram dramaticamente o 
interesse no estudo de problemas dinâmicos em grafos.  O estudo da conexidade 
de tais redes é uma das primeiras etapas no entendimento da dinâmica da rede, 
por isso este estudo tem um papel central nessa área. 

O tema escolhido para este projeto serve como motivação para o estudo de tópicos 
centrais de áreas importantes da Ciência da Computação, como projeto e análise 
de algoritmos e estruturas de dados. 

%%% síntese da bibliografia fundamental
% O problema específico escolhido neste projeto é importante e rico, 
% como demonstra o levantamento bibliográfico inicial.  
% O problema é discutido em diversos livros~\cite{Gusfield97,SetubalM97,Vazirani01} e
% em uma série de artigos, alguns clássicos~\cite{BlumJLTY94,Li90,Turner89} e
% vários outros, alguns deles mais
% recentes~\cite{ArmenS95,ArmenS95a,ArmenS98,Bongartz01,CzumajGPR97,KaplanS05,KosarajuPS94,Sweedyk00,TengY97,WeinardS06}.}

%%% resumo do objetivo
% Esta iniciação científica visa o estudo de algoritmos de aproximação e
% resultados de complexidade para o SCS. Especial ênfase será dada aos
% resultados referentes à conjectura sobre o algoritmo guloso para o
% SCS.

%%% o aluno, o BCC e o seu histórico escolar
O Daniel é um aluno do terceiro ano do Bacharelado em Ciência da Computação (BCC) 
do IME-USP. Embora suas notas no primeiro ano não sejam tão boas, levando-o a 
uma média $7{,}0$ nas disciplinas que cursou, o Daniel não obteve nenhuma reprovação, 
e melhorou significativamente seu aproveitamento no segundo ano, obtendo notas todas
acima de $7{,}0$.  O seu interesse em fazer iniciação científica demonstra esse 
amadurecimento e certamente vai consolidar ainda mais essa melhora do seu 
aproveitamento no curso. 

%%% ele já fez disciplinas importantes 
As disciplinas do BCC que já foram concluídas pelo Daniel são sufi\-cien\-tes 
para levar adiante este projeto.  Dentre as disciplinas que ele cursou, 
destacamos {\small\sc MAC0121 Estruturas de Dados e Algoritmos I} e 
{\small\sc MAC0323 Estruturas de Dados e Algoritmos II}, que dão 
uma base essencial para o desenvolvimento desse projeto. 